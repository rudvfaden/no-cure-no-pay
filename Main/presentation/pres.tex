% -*- program: xelatex -*-
\documentclass[compress, handout]{beamer}
\usetheme{m}
\setbeamercovered{transparent}
\usepackage{commath}
\usepackage{mathtools}
\usepackage{cleveref}
% bibliography
\RequirePackage{csquotes}
\emergencystretch=1em
\hfuzz=5.002pt 
\RequirePackage[%
backend=biber,                       % use BibTeX
style=chicago-authordate,
url=false, isbn=false, doi=false
]{biblatex}
\addbibresource{../Remote.bib}
\usepackage{silence}
\WarningFilter{biblatex}{Patching footnotes failed}

%% author and title 
\title{No cure no pay --- contracts with limited liability}
\subtitle{}
\date{\today}
\author{Rud Faden}
\institute{University of Copenhagen}

\hypersetup{
pdfauthor = {Rud Faden: rudfaden@gmail.com},
pdfsubject = {No cure no pay --- contracts with limited liability},
pdfkeywords = {contingent contracts, limited liability, health care},
pdfcreator = {LaTeXing}
}

\begin{document}

\maketitle

\section*{Introduction \& Motivation}
\begin{frame}{Introduction}
  \begin{itemize}[<+- | alert@+>]
    \item Health care cost are rising --- there is an interest in increasing productivity
    \item Payment schemes should be constructed to increase output
    \item But most focus seems to be on motivating the ``health organization''
    \item Usually contracts are linear --- per item --- fees or fixed wages
    \item I show that the optimal contract is a non-linear, contingent claim contract
  \end{itemize}
\end{frame}

\begin{frame}[c]{Contingent contracts}
 \begin{itemize}[<+- | alert@+>]
   \item Contingent contracts: payment is conditional on outcome
   \item Well known from real estate and US lawyers
   \item \cite{Lambert2005No} shows that shows that in the optimal contingent state contract, the physician pays a penalty fees to the principal
   \item In many real life situations, such a contract may no be feasible
   \item Therefore i assume that the optimal contract is subject to limited liability
 \end{itemize}
\end{frame}


\begin{frame}[c]{The plan}
  \begin{itemize}[<+- | alert@+>]
      \item Assume that both physician and principal are risk neutral
      \item Introduce the model
      \item show that a fixed wage is optimal when effort is observed, and in-optimal if not
      \item Show that the optimal contract is non-linear
      \item Future work
    \end{itemize}  
\end{frame}

\section{The Model}
    
\begin{frame}[c]{Environment I}
\begin{itemize}[<+- | alert@+>]
  \item The physician is employed by a health organization (hospital/municipal)
  \item The physicians work is measured by observable output $y$
  \item $y$ is random, with density $g(y|e)$ where $e$ is the physicians effort
  \item It is assume that $g(\cdot)$ has a monotone likelihood ratio property
  \[
    \frac{\partial}{\partial y}\left(\frac{g_e(y|e)}{g(y|e)}\right)>0
  \]
  I.e.\ $y$ and $e$ are ``complements''
  \item Effort comes at a cost $C(e)$
\end{itemize}
\end{frame}

\begin{frame}[c]{Environment II}
\begin{itemize}[<+- | alert@+>]
  \item The physicians payment is given by $r=R(y,e)$
  \item The physician has a utility function $u(r,y)$, which depends on payment and output
  \item Dependence on output allows for a ``caring'' physician
  \item The hospital has a budgetary income $v(y)$ from which it pays the physician
\end{itemize}
\end{frame}

\begin{frame}[c]{The incentive problem}
  The main problem is to find a contract $r=R(y,e)$, with the property that expected utility of the physician
  \begin{align}
       U(R,e)=\int u(R(y,e),y)g(y|e)\dif y-C(e) \label{eq:phys-util}
  \end{align}
  cannot be increased by a change in $R(y,e)$, without decreasing the expected net income of the hospital
  \[
    V(R,e)=\int \left[y-R(y,e)\right] g(y|e)\dif y
  \]
  subject to the \emph{incentive constraint} that effort in \cref{eq:phys-util} insures maximal utility
\end{frame}

\begin{frame}[c]{Limited Liability}
  \begin{itemize}[<+- | alert@+>]
    \item A key feature of the model is limited liability
    \item Implies that the physician cannot be forced to pay for bad outcome and the hospital cannot be forced to pay more than the value of output
    \item Formally $0\leq R(y) \leq y$
    \item Limited liability reduces the set of contracts
    \item E.g.\ it rules out a \citeauthor{Mirrlees1974Notes} forcing contract
  \end{itemize}
\end{frame}

\begin{frame}[c]{The Formal Problem}
  \begin{subequations}
\label{eq:khun-tucker-general}
\begin{align}
    \max_{R,e} & \int_0^{\infty}\left[u(R(y,e))+\delta y\right]g(y|e)\dif y-C(e)\label{subeq:khun-tucker1-general} \\
    \text{s.t. }    & \int_{0}^{\infty} v(y-R(y,e))g(y|e)\dif y\geq y_L^0 \label{subeq:khun-tucker2-general} \\
                    & E[u(R,e)|e]\leq E[u(R,e^*)|e] \label{subeq:khun-tucker3-general}\\
                    & 0\leq R(y,e)\leq y \label{subeq:khun-tucker4-general}
\end{align}
\end{subequations}
where \cref{subeq:khun-tucker1-general,subeq:khun-tucker2-general} are the physicians and hospitals utility. \cref{subeq:khun-tucker3-general} is the physicians \emph{incentive constraint} and \cref{subeq:khun-tucker4-general} is the limited liability constraint
\end{frame}

\begin{frame}[c]{First-Best and the In-optimality of Fixed Wage}
  \begin{itemize}[<+- | alert@+>]
    \item When effort is observable the problem becomes
\begin{align*}
    \max_{R,e} & \int_0^{\infty}\left[u(R(y,e))+\delta y\right]g(y|e)\dif y-C(e)\\
    \text{s.t. } & \int_{0}^{\infty} v(y-R(y,e))g(y|e)\dif y\geq y_L^0 
\end{align*}
    \item Using point-wise optimization the optimal contract is implicitly given by 
    \[
      \frac{v'(y-R(y,e))}{u'(R(y,e))}=\frac{1}{\lambda}
    \]
    I.e.\ a fixed wage
    \item This is the \emph{first-best} solution
  \end{itemize}
\end{frame}

\begin{frame}[c]{First-Best and the In-optimality of Fixed Wage}
  \begin{itemize}[<+- | alert@+>]
    \item In the following I assume that both parties are risk neutral and the utility is additive and separable
    \item I.e.\ $u(\cdot)=R(y,e)+\delta y$ and $v(\cdot)=y-R(y,e)$
  \end{itemize}
\end{frame}

% bibliography
\begin{frame}[c]{Bibliography}
\printbibliography%
  
\end{frame}
\end{document}
